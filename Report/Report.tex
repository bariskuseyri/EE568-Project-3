\documentclass[a4paper, 11pt, titlepage]{article}
\usepackage{graphicx}
\usepackage{subcaption}
%\usepackage{nath}
%\usepackage{gensymb}
\usepackage{indentfirst}

\usepackage{tikz}
\usepackage{circuitikz}

\addtolength{\hoffset}{-1.5cm}
\addtolength{\textwidth}{3cm}

\begin{document}
\title{EE568 Project 3: PM Motor Comparison Analysis}
\author{Baris Kuseyri}
\date{\today}
\maketitle

\pagenumbering{arabic}
\tableofcontents
\newpage

\section{Introduction}
\label{sec:intro}

In this assignment you are going to design several surface-mount PM machines with the following constant parameters:

Surface-mount permanent magnet (SPM) topology is kind of a machine where PMs mounted on the rotor surface, facing to the airgap [\ref{hanselman}].

This project aims to prepare a study similar to what was done by W. L. Soong in his case study on Ferrite versus NdFeB. In his study, Soong replaces rare earth NdFeB magnets with ferrite magnets in an SPM machine. Later, he increases the thickness of the magnets and equalizes the flux density levels on stator tooth and back iron region of the machine with NdFeB magnets and ferrite magnets. Unlike Soong's study, this project sets the stator outer diameter constant.

First, this project designs a SPM in part \ref{sec:Q1}, to analyse the magnetic loading of a machine. Then, in part \ref{sec:Q2}, it determines several machine parameters. The project continues by analysing electric loading of the machine, and the resulting average tangential stress, total force, torque and power output.


This project aims to optimize a SPM machine, in which neodymium magnets. 

This project 

The starting values are given in Table \ref{table:machineParameters}, below.

\begin{table}[ht]
	\begin{center}
		\begin{tabular}{c|c|c|c}
			 & symbol & unit & value \\
			\hline
			number of phases & $m$ & & 3 \\
			number of poles & $p$ & & 4 \\
			motor axial length & $l_m$ & mm & 100 \\ 
			air-gap clearance & $\delta_g$ & mm & 1 \\
			magnet to pole pitch ratio & & & 0.8 \\
			magnet radial thickness & $t_m$ & mm & 4 \\
			\hline
		\end{tabular}
	\end{center}
	\caption{Machine Parameters}
	\label{table:machineParameters}
\end{table}
	


\section{Q1- Magnetic Loading}
\label{sec:Q1}

This project starts by analysing the magnetic loading of an SPM machine. The stator is assumed to be solid for this part. Additional to the parameters given in Table \ref{table:machineParameters}, some extra values are set as starting values. Parameters on rare earth magnet are given in Table \ref{fig:PMParameters} and rotor diameter is given in Table \ref{table:rotorParameter}.





\begin{table}[h]
	\begin{center}
		\begin{tabular}{c|c|c|c}
			 & symbol & units & value \\
			\hline
			magnet type & & & NdFeB N42 \\
			shape & & & radial \\ 
			relative permeability & $\mu_r$ & & 1.05 \\
			coercivity & $H_c$ & $A/m$ & 994529 \\
			\hline
		\end{tabular}
	\end{center}
	\caption{Permanent Magnet Parameters}
	\label{fig:PMParameters}
\end{table}


\begin{table}[h]
	\begin{center}
		\begin{tabular}{c|c|c|c}
			 & symbol & unit & value \\
			\hline
			rotor diameter & $D_r$ & mm & 100 \\
			\hline
		\end{tabular}
	\end{center}
	\caption{Rotor Dimensions}
	\label{table:rotorParameter}
\end{table}




\subsection{a. Magnetic Equivalent Circuit, Magnet Load Line and Peak Air-gap Flux Density}

Magnetic equivalent circuit for one pole-pair is shown in Fig. \ref{fig:magneticCircuit}. Here, each PM is modelled as an MMF source, meaning it's equivalent model is composed of a voltage source with a value of $F=\phi R_m$, and a resistance $R_m$ connected in series.

Starting with PM \#1, the flux $\phi$ goes through the airgap, modelled as $R_g$, and splits into two components at the stator back iron. Each half of the flux goes through the back iron, in opposite directions. Here, only one pole-pair is shown, so only one half-of-the-flux is described further. This half flux component then merges with, again what is another half flux component, and travels through the airgap, modelled again as $R_g$. Then, it travels through PM \#2, and splits into two. One of the components merges with another half flux components and goes through PM \#1. Hence, the cycle is completed.

\begin{figure}[h]
	\begin{center}
		\begin{circuitikz}
			\draw (0,0)
			to[short,i_=$\frac{\phi}{2}$,*-*] (0,2)
			to[short,i_=$\phi$,*-*] (1,2)
			to[battery,l^=$F$, invert,*-] (2,2)
			to[R=$R_m$,i_=$\phi$,-*] (5,2)
			to[R=$R_g$] (7,2)
			to[short,i_=$\phi$] (8,2)
			to[short,i_=$\frac{\phi}{2}$,*-*] (8,0)
			to[short,i_=$\phi$] (7,0)
			to[R=$R_g$] (5,0)
			to[battery,l^=$F$, invert,*-] (4,0)
			to[R=$R_m$,i_=$\phi$,-*] (1,0)
			to[short,i_=$\phi$] (0,0);
			\draw (0,4)
			to[short,i_=$\frac{\phi}{2}$] (0,2);
			\draw (8,2)
			to[short,i_=$\frac{\phi}{2}$] (8,4);
			\draw (8,-2)
			to[short,i_=$\frac{\phi}{2}$] (8,0);
			\draw (0,0)
			to[short,i_=$\frac{\phi}{2}$] (0,-2);
			\draw[black, thick] (1,1.5) rectangle (5,3.5)
			node[pos=0.5, above=10pt]{PM \#1};
			\draw[black, thick] (1,-1) rectangle (5,1)
			node[pos=0.5, above=10pt]{PM \#2};
		\end{circuitikz}
	\caption{magnetic equivalent circuit for one pole-pair}
	\label{fig:magneticCircuit}
	\end{center}
\end{figure}


\subsection{b. Magnetic Loading}

Magnetic loading of a machine refers to the average airgao flux density over a pole. This value can be calculated by

\begin{equation}
	\bar{B}=\frac{p\phi_p}{\pi D_rl}
	\label{eq:specificMagneticLoading}
\end{equation}

where $\bar{B}$ is specific magnetic loading, $p$ is the number of poles, $\phi_p$ is flux per pole, $D_r$ is rotor diameter, and $l$ is the axial length of the machine. Equation \ref{eq:specificMagneticLoading} interprets specific magnetic loading $\bar{B}$ as total flux going out of the rotor surface divided by total rotor surface area.

To find out $\phi_p$, machine's magnetic circuit is needed to be analyzed. The magnetic circuit diagram can be seen in Fig. \ref{fig:magneticCircuit}. 

The magnetic circuit is composed of 2 NdFeB 42 permanent magnets (PM) and 2 instances of airgaps. To analyse this circuit, PM remanence flux density $B_r$ information is required. At this point, relative permeability $\mu_r$ and coercivity $H_c$ of NdFeB 42 PM is given as a part of the problem, which can be seen in Table \ref{fig:PMParameters}. From here; remanence flux density can be found by the following relation:

\begin{equation}
	B_r = \mu_0\mu_r\times H_c 
	\label{label:remanenceFluxDensity}
\end{equation}

where, $\mu_0$ is the permeability of air or vacuum ($\mu_0=4\pi \times 10^{-7}A/m$). Hence, magnet remanence flux density  is $B_r=1.31 T$. Now,

Neglecting leakage flux:

\begin{equation}
	B_mA_m = B_gA_g
\end{equation}

Assuming infinitely permeable core ($\mu_c=\infty$, where $\mu_c$ is the core permeability)

\begin{eqnarray}
	H_ml_m + H_gl_g &=& 0 \\
	H_ml_m &=& -H_gl_g
\end{eqnarray}

Now, in the airgap

\begin{eqnarray}
	B_g &=& \mu_oH_g \\
	B_mA_m &=& \mu_0H_gA_g=-\mu_0H_mA_g\frac{l_m}{l_g} \\
	\frac{B_m}{H_m} &=& -\frac{\mu_0A_g}{A_m}\frac{l_m}{l_g}
	\label{label:loadLine}
\end{eqnarray}

This is the equation of the so-called load-line of the magnetic circuit. 

For a material with a linear demagnetisation characteristic:

\begin{eqnarray}
	B_m &=& B_r + \mu_0\mu_rH_m \\
	H_m &=& \frac{B_m-B_r}{\mu_0\mu_r} \\
	B_m &=& -\frac{A_gl_m}{A_ml_g\mu_r}(B_m-B_r) \\
	B_m(\frac{A_gl_m}{A_ml_g\mu_r}+1) &=& \frac{A_gl_m}{A_ml_g\mu_r}B_r \\
	B_m &=& \frac{\frac{A_gl_m}{A_ml_g\mu_r}}{\frac{A_gl_m}{A_ml_g\mu_r}+1}B_r \\
	B_m &=& \frac{B_r}{1+\frac{A_ml_g\mu_r}{A_gl_m}}
	\label{label:demagnetizationCharacteristics}
\end{eqnarray}

Assuming $A_g = A_m$, the above equation simplifies to

\begin{equation}
	B_g = B_m = \frac{B_r}{1+\mu_r\frac{l_g}{l_m}}
\end{equation}

where $B_g$ is the airgap flux density, $l_g$ is the airgap clearance and $l_m$ is the magnet thickness. Here, $B_g$ value corresponds to the peak airgap flux density $\hat{B}_g$. The average airgap flux density corresponds to the $RMS$ value of peak airgap flux density

\begin{equation}
	B_{avg} = \frac{1}{\sqrt{2}}\hat{B}_g
	\label{label:}
\end{equation}

which corresponds to specific magnetic loading $\bar{B}$ of the machine.


\subsection{c. FEA Results}

FEMM is used as FEA software.

\begin{figure}[h]
	\includegraphics[width=\textwidth]{fluxDensity.png}
	\caption{Flux Density Plot}
	\label{fig:Bplot}
\end{figure}

\begin{figure}[h]
	\includegraphics[width=\textwidth]{airgapFluxDensity.png}
	\caption{Airgap Flux Density Distribution}
	\label{fig:airgapBplot}
\end{figure}


\section{Q2- Electrical Loading \& Machine Sizing}
\label{sec:Q2}

\subsection{a. Number of Slots}

Number of slots for this machine is determined to be 12.


\subsection{b. AWG Cable}

\begin{table}[h]
	\begin{center}
		\begin{tabular}{c|c|c|c}
			 & symbol & units & value \\
			\hline
			max. current density & $J$ & $A/mm^2$ & 5 \\
			coil current & $I$ & A & 2.5 \\ 
			fill factor & $K_p$ & & 0.6 \\
			\hline
		\end{tabular}
	\end{center}
	\caption{Permanent Magnet Parameters}
	\label{fig:PMParameters}
\end{table}


If the maximum current density $\hat{J}$ is set to be $5 A/mm^2$ and one conductor is carrying $2.5A$, then the maximum number of conductors there can be in $1mm^2$ is 2, and the total cross-section area of 2 conductors can not be below $1mm^2$. Therefore, the suitable AWG cable chosen for this project is AWG 20. The characteristics of AWG20 cable can be seen in Table \ref{tab:AWG20}, below.

\begin{table}[h]
	\begin{center}
		\begin{tabular}{c|c|c|c}
			AWG & diameter [$mm$] & area [$mm^2$] & resistance/length [m$\Omega$/m] \\
			\hline
			20 &  0.812 & 0.518 & 33.31 \\
			\hline
		\end{tabular}
	\end{center}
	\caption{AWG 20 characteristics}
	\label{tab:AWG20}
\end{table}


As can be seen in Table \ref{tab:AWG20}, the cross-section area of AWG 20 cable is $0.518mm^2$, which corresponds to $J=4.83A/mm^2$, below the maximum current density value given for this project. Using any cable with a cross-section area below $0.500mm^2$ with a coil current of $I=2.5A$ would exceed this current density $J$ limitation.


\subsection{c. Slot Height, Number of Coils per Slot, and Back-core Thickness}

This part starts by choosing a slot ratio $d$ for the machine. This choice is done in the following fashion. Teeth shape is determined to be rectangular. Then, the slot area is calculated by



where $A_s$ is slot area, $N_s$ is the number of slots, $r_{so}$ is the slot outer radius, $r_{si}$ is the slot inner or stator inner radius, and $A_t$ is the area of tooth.

First, tooth to slot opening ratio is assumed to be 1 to 1. Then, 

\begin{equation}
	\tau_{teeth} = \tau_{slot} = \frac{2\pi r_{si}}{2N_s}
\end{equation}

where $\tau_{teeth}$ is the tooth thickness and $\tau_{slot}$ is the slot opening. 

\begin{equation}
	h_t = \frac{r_{si}}{d}-r_{si}
\end{equation}

where $h_t$ is tooth height. Then, tooth area is calculated by

\begin{equation}
	A_t = h_t \tau_{teeth}
\end{equation}

\begin{equation}
	N_sA_s = (\pi r_{so}^2 - \pi r_{si}^2)-N_sA_t
\end{equation}

Hence, the slot area $A_s$ is calculated.

Then, the number of conductors to be fit in a slot is determined by

\begin{equation}
	N_{cond} = \frac{A_sK_p}{A_{cond}}
	\label{eq:numberOfConductorsPerSlot}
\end{equation}

Hence, the number of conductors to be fit in one slot is $N_{cond}=999$. 


\subsection{d. Electrical Loading}


Now that the number of conductors per slot is determined, the electric loading of the machine can be calculated.

\begin{equation}
	\hat{A} = \frac{N_sN_{cond}I}{2\pi r_{si}}\frac{1}{\sqrt{2}}
	\label{eq:specificElectricLoading}
\end{equation}

\subsection{e. Average Tangential Stress \& Total Force}

\begin{equation}
	l'=l+2l_g
	\label{eq:effectiveAxialLength}
\end{equation}


\begin{eqnarray}
	\sigma_{tan} = \frac{\bar{A}\hat{B}}{\sqrt{2}} \\
	F = 2\sigma_{tan}\pi r_{r}l' \\
	T = Fr_{r}= 2\sigma_{tan}\pi r^2_{r}l'
	\label{eq:specificElectricLoading}
\end{eqnarray}

\subsection{f. Expected Power Output}

\begin{equation}
	P=T\times\omega
\end{equation}


\section{Q3- Comparison \& Optimization}



\subsection{a. Optimum Rotor Diameter}

\subsection{b. Replacing NdFeB with Ferrite}

\subsection{c. Optimizing Ferrite Machine}

\section{Conclusion}




















\newpage

\bibliography{bibliography} 
\bibliographystyle{plain}

\end{document}