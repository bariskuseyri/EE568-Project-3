\documentclass[a4paper, 11pt, titlepage]{article}
\usepackage{graphicx}
\usepackage{subcaption}
%\usepackage{nath}
%\usepackage{gensymb}
\usepackage{indentfirst}

\usepackage{tikz}
\usepackage{circuitikz}

\begin{document}
\title{EE568 Project 3: PM Motor Comparison Analysis}
\author{Baris Kuseyri}
\date{\today}
\maketitle

\pagenumbering{arabic}
\tableofcontents
\newpage

\section{Introduction}

\begin{table}[ht]
	\begin{center}
		\begin{tabular}{c|c|c|c}
			 & symbol & unit & value \\
			\hline
			number of phases & $m$ & & 3 \\
			number of poles & $p$ & & 4 \\
			motor axial length & $l_m$ & mm & 100 \\ 
			air-gap clearance & $\delta_g$ & mm & 1 \\
			magnet to pole pitch ratio & & & 0.8 \\
			rotor diameter & $D_r$ & mm & 100 \\
			magnet radial thickness & $t_m$ & mm & 4 \\
			\hline
		\end{tabular}
	\end{center}
	\caption{Machine Parameters}
	\label{table:machineParameters}
\end{table}
	


\section{Q1- Magnetic Loading}

\begin{figure}[h!]
	\begin{center}
		\begin{circuitikz}
			\draw (0,0)
			to[short] (0,0.5)
			to[battery,l^=$F$, invert] (0,2)
			to[R=$R_m$] (0,3.5)
			to[short] (0,4)
			to[short,i_=$\phi$] (1,4)
			to[R=$R_g$] (3,4)
			to[short] (4,4)
			to[short] (4,3.5)
			to[battery,l^=$F$, invert] (4,2)
			to[R=$R_m$] (4,0.5)
			to[short] (4,0)
			to[R=$R_g$] (0,0);
		\end{circuitikz}
	\caption{magnetic equivalent circuit for one pole-pair}
	\label{fig:magneticCircuit}
	\end{center}
\end{figure}

\begin{table}[ht]
	\begin{center}
		\begin{tabular}{c|c|c|c}
			 & symbol & units & value \\
			\hline
			magnet type & & & NdFeB N42 \\
			shape & & & radial \\ 
			relative permeability & $\mu_r$ & & 1.05 \\
			coercivity & $H_c$ & $A/m$ & 994529 \\
			\hline
		\end{tabular}
	\end{center}
	\caption{Permanent Magnet Parameters}
	\label{fig:PMParameters}
\end{table}



\begin{equation}
	B_r = \mu_0\mu_r\times H_c 
	\label{label:remanenceFluxDensity}
\end{equation}

\begin{eqnarray}
	B_mA_m & = & B_gA_g \\
	H_ml_m & = & -H_gl_g \\
	B_g &=& \mu_oH_g \\
	B_mA_m &=& \mu_0H_gA_g=-\mu_0H_mA_g\frac{l_m}{l_g} \\
	\frac{B_m}{H_m} &=& -\frac{\mu_0A_g}{A_m}\frac{l_m}{l_g}
	\label{label:loadLine}
\end{eqnarray}

This is the equation of the so-called load-line of the magnetic circuit. 

For a material with a linear demagnetisation characteristic:

\begin{eqnarray}
	B_m = B_r + \mu_0\mu_rH_m \\
	H_m = \frac{B_m-B_r}{\mu_0\mu_r} \\
	B_m=-\frac{A_gl_m}{A_ml_g\mu_r}(B_m-B_r) \\
	B_m(\frac{A_gl_m}{A_ml_g\mu_r}+1)=\frac{A_gl_m}{A_ml_g\mu_r}B_r \\
	B_m = \frac{\frac{A_gl_m}{A_ml_g\mu_r}}{\frac{A_gl_m}{A_ml_g\mu_r}+1}B_r \\
	B_m = \frac{B_r}{a+\frac{A_ml_g\mu_r}{A_gl_m}}
	\label{label:demagnetizationCharacteristics}
\end{eqnarray}

\begin{equation}
	B_{avg} = \frac{1}{\sqrt{2}}\hat{B_g}
\end{equation}

\begin{figure}[b]{1.00\textwidth}
	%\includegraphics[width=\textwidth]{flux_density.bmp}
	\caption{}
	\label{subfig:Q1_dist_factor}
\end{figure}

\subsection{a}

\subsection{b Magnetic Loading}

\subsection{c FEA Results}


\section{Q2- Electrical Loading \& Machine Sizing}


\subsection{a}

\subsection{b}

\subsection{c}

\subsection{d}

\subsection{e}

\subsection{f}


\section{Q3- Comparison \& Optimization}



\subsection{a}

\subsection{b}

\subsection{c}




















\newpage

\bibliography{bibliography} 
\bibliographystyle{plain}

\end{document}